The statistical software used for the last several years in high energy physics is facing scalability challenges.
In addition to processing speed, which is being addressed with GPU-based fitting approaches, we also face memory limitations as the combined statistical models grow in size.
Thus, it is critical to investigate more distributed models for these computations.\\

The rapid development of software libraries for numerical computations through data flow graphs (e.g., \code{TensorFlow} and \code{Theano}) has led to a fundamental change of paradigm in machine learning software.
These libraries are designed around the concept that a numerical program can often equivalently be expressed as a graph, where nodes represent mathematical operations and edges represent the data communicated between them.
While originally developed for the purpose of deep learning research, they are general enough to be applicable in a wide variety of other domains.
Under the mentorship of Gilles Louppe and Vince Croft, Matthew Feickert will conduct a feasibility study to answer whether statistical models used in particle physics could equivalently be expressed as computational graphs, assess their capabilities and limits, and determine how those frameworks would scale in terms of data and model parallelism.
